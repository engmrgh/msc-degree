مقالات بررسی شده از مجموعه داده‌های مختلفی استفاده کرده‌اند که در این جا به صورت
خلاصه جزئیات آن‌ها بررسی می‌شود.

\section{مجموعه داده روزنامه نیویورک تایمز}

مجموعه داده نیویورک تایمز \LTRfootnote{NYT-dataset} در سال 2010 توسط ریدال \cite{nyt-dataset} ارائه شد.
این مجموعه داده که شامل ۵۳ رابطه مختلف است،
یکی از مجموعه داده‌های پرکاربرد در حوزه استخراج رابطه بوده و توسط مقالات مختلف استفاده
شده است. برای ساخت این مجموعه داده از روش نظارت از راه دور استفاده شده و جملات روزنامه نیویورک‌تایمز
با استفاده از پایگاه دانش فری‌بیس\LTRfootnote{freebase} برچسب‌گذاری شده است.
این مجموعه داده شامل ۵۲۲۶۱۱ جمله برای آموزش و ۱۷۲۴۴۸ جمله برای آزمون است.

\section{مجموعه داده \lr{SemEval-2010}}

مجموعه داده \lr{SemEval-2010} نیز در سال ۲۰۱۰ ارائه شده است \cite{hendrickx-etal-2010-semeval}.
این مجموعه داده نسبت به مجموعه داده نیویورک تایمز کوچک‌تر بوده و شامل ۱۹ رابطه مختلف است.
این مجموعه داده شامل 10717 جمله است که از این تعداد 8000 جمله برای آموزش و باقی برای آزمون
استفاده می‌شود.

\section{مجموعه داده \lr{SemEval-2018}}

این مجموعه داده در سال ۲۰۱۸ با برچسب‌گذاری داده‌های گزارش‌ حملات امینتی ارائه شد \cite{phandi-etal-2018-semeval}.
تعداد جملات برچسب‌گذاری شده این مجموعه داده مشابه مجموعه داده \lr{SemEval-2010} بوده اما تعداد
رابطه‌های آن نسبت به مجموعه داده \lr{SemEval-2010} بسیار کم‌تر است. این مجموعه داده شامل
10182 جمله است که از این تعداد ۸۹۱۹ جمله برای آموزش مدل و باقی 1263 جمله برای آزمون استفاده می‌شود.
این مجموعه داده شامل ۴ رابطه مختلف است.

\section{مجموعه داده \lr{UW}}

این مجموعه داده در سال ۲۰۱۶ توسط لیو\cite{liu-etal-2016-effective} ارائه شده است. این مجموعه داده شامل ۵ رابطه
بوده اما تعداد جملاتی که برای هر رابطه ارائه کرده است در اندازه مجموعه داده نیویورک تایمز است. این مجموعه داده دارای
حدود ۵۰۰ هزار جمله برای آموزش و حدود ۳۷۲۴ جمله برای آزمایش مدل است.