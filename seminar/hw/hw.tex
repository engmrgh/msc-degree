\documentclass[14pt,a4]{article}

\usepackage{amsmath}
\usepackage{enumitem}
\usepackage{graphicx}
\usepackage{xcolor}
\usepackage{subcaption}
\usepackage{xepersian}

\settextfont{XB Kayhan}
\setlatinsansfont{Times Newer Roman}

\title{\vspace{-4cm} \textbf{پاسخ تکلیف مقاله‌نویسی درس سمینار}}
\author{محمدرضا غفرانی  ۴۰۰۱۳۱۰۷۶}
\date{\today}

\begin{document}

\maketitle

\section{عنوان مقاله}

\subsection{نقاط ضعف}

\begin{enumerate}
    \item عنوان مقاله هدف از انجام پژوهش را بیان نمی‌کند.
\end{enumerate}

\subsection{نقاط قوت}

\begin{enumerate}
    \item عنوان طولانی نیست و قسمت اصلی عنوان دارای سه کلمه است.
    \item در عنوان از عبارات مختصر و فرمول استفاده نشده است.
\end{enumerate}

\section{چکیده}

\subsection{نقاط ضعف}

\begin{enumerate}
    \item در چکیده مقاله به دقتی که در طی پژوهش رسیده‌اند اشاره نشده است.
    \item به جمع‌آوری داده‌ها اشاره نشده است.
    \item به روش انجام کار اشاره نشده است.
\end{enumerate}

\subsection{نقاط قوت}

\begin{enumerate}
    \item همانطور که می‌دانیم طول چکیده باید بین 50 تا 300 کلمه باشد. از آن‌جا که چکیده این مقاله دارای 106 کلمه است
    بنابراین این مطلب رعایت شده است.
    \item هدف از انجام گزارش بیان شده است.
    \item شیوه ارزیابی مدل بیان شده است.
    \item انگیزه کار روی تحقیق بیان شده است.
    \item به کاربرد حاصل از مدل اشاره شده است.
\end{enumerate}

\section{مقدمه}

\subsection{نقاط ضعف}

\begin{enumerate}
    \item در آخرین پاراگراف، مطالب موجود در بخش‌های بعدی بیان نشده است.
\end{enumerate}

\subsection{نقاط قوت}

\begin{enumerate}
    \item نویسنده به یکباره سراغ بحث اصلی نرفته است و مقدمه را با جنبه‌های عمومی‌تر موضوع آغاز کرده است.
    \item به کار‌های اصلی انجام شده در گذشته اشاره شده و نقاط ضعف و قوت آن‌ها بیان شده است. البته
    بیان نقاط ضعف در متن پررنگ‌تر است ولی در کلیت بیان مطلب به نقاط قوت نیز اشاره شده است.
    \item روش انجام کار، شیوه انجام پژوهش و دستاورد‌های پژوهش در مقدمه توضیح داده شده است.
\end{enumerate}

\section{مرور سوابق}

\subsection{نقاط ضعف}

\begin{enumerate}
    \item در بعضی از ارجاع‌های صورت گرفته خود ارجاع بخشی از جمله است.
\end{enumerate}

\subsection{نقاط قوت}

\begin{enumerate}
    \item هر کار مختصرا بررسی شده و نقاط قوت و ضعف آن بیان شده است.
    \item غالب ارجاع‌های صورت گرفته به مقالاتی هستند که اخیرا چاپ شده‌اند در نتیجه
    می‌توان بیان کرد که پژوهشگر کار‌هایی را که اخیرا در زمینه مورد بحث انجام شده است را به خوبی
    مطالعه کرده است.
    \item مطلب مورد بحث از جنبه‌های مختلف مورد بحث و ارزیابی قرار گرفته است.
\end{enumerate}

\section{روش پیشنهادی}

\subsection{نقاط ضعف}

\begin{enumerate}
    \item در یکی از قسمت‌ها ویرگول فراموش شده بود.
    \item در این پژوهش برای مقایسه مدل توسعه‌یافته جدید با مدل‌های موجود، مدل‌های موجود مجددا پیاده‌سازی شده بودند.
    اما از مدل‌های پیشین بهترین آن‌ها برای پیاده‌سازی انتخاب نشده بود بلکه به ادعای نویسنده ساده‌ترین این
    مدل‌ها پیاده‌سازی شده بودند.
    \item برای هیچ یک از صفحات شماره صفحه زده نشده است.
    \item عنوان و شماره جدولِ جدول‌های ۱ و ۲ در بالای جدول قرار نگرفته است.
\end{enumerate}

\subsection{نقاط قوت}

\begin{enumerate}
    \item مقاله دارای جزئیات پیاده‌سازی خوبی است که در بخش‌هایی با رنگ سبز متمایز شده است.
    \item پژوهشگران این مقاله از تجربیات برآمده از سایر مقاله‌ها در پژوهش‌های خود استفاده کرده بودند.
    \item در این بخش در ارجاع‌های صورت گرفته خود ارجاع بخشی از متن جمله نیست.
\end{enumerate}

\section{نتایج تجربی}

\subsection{نقاط ضعف}

\begin{enumerate}
    \item در این بخش عنوان و شماره جدول‌های ۳ و ۴ به اشتباه در پایین جدول آورده شده است.
    \item جدول ۴ ابتدا در مقاله آورده شده و سپس به آن در متن ارجاع داده شده است.
    \item بخشی از شیوه ارزیابی مدل در قسمت روش پیشنهادی و بخشی دیگر در ذیل عنوان «نتایج تجربی» آورده شده است.
    بهتر بود که همه‌ی مطالب متناظر ارزیابی در ذیل بخش »نتایج تجربی« و ذیل همین عنوان آورده می‌شد.
    \item خواننده نمی‌تواند با مراجعه مستقیم به جدول متوجه بشود که معیار Fluency، Engagingness و Consistency
    از بین 1 تا 5 است و در نتیجه عدد 4.31 برای او بی‌معنی می‌شود.
\end{enumerate}

\subsection{نقاط قوت}

\begin{enumerate}
    \item در مقاله علاوه بر موضوع مورد بحث به سایر جنبه‌های کار نیز دقت شده است.
    برای مثال در این بخش علاوه بر ساخت مدلی که توانایی چت کردن دارد سعی شده است شخصیت افراد حاضر در چت تعیین شود.
    \item در نهایت پیشنهاد‌هایی نیز برای کار‌های آتی که می‌تواند در این خصوص انجام شود اشاره شده است.
    \item نتایج به تفکیک مدل‌های اعمال شده و روش‌های ارزیابی مختلف ارائه شده است.
    \item برای معتبرتر کردن نتایج ارائه شده از دو روش برای ارزیابی مدل استفاده شده است.
    یکی روش‌های ارزیابی ریاضی و دیگری نظر کاربران در هنگام استفاده از مدل.
    \item تمامی جدول‌های ارائه شده از لحاظ ساختار یک شکل و یک دست هستند.
\end{enumerate}

\section{بحث}

\subsection{نقاط ضعف}

\begin{enumerate}
    \item عنوان «بحث» در کنار عنوان «نتیجه‌گیری» آورده شده است در حالی که بحث بر روی نتایج در بخش
    نتایج تجربی انجام شده است.
\end{enumerate}

\subsection{نقاط قوت}

\begin{enumerate}
    \item نتایج حاصل از اعمال روش‌های پیشنهادی از چندین زاویه بررسی و تحلیل شده است.
    \item نتایج حاصل شده به دقت تحلیل شده و
    علل ضعف و قوت آن بررسی شده است. بدین شکل نیست که تاکید زیادی روی نتایج مثبت شده باشد.
\end{enumerate}

\section{نتیجه‌گیری}

\subsection{نقاط ضعف}

\begin{enumerate}
    \item انتظار داشتیم که در ابتدا توصیفی کلی از مسئله مورد بحث ارائه شود و سپس به روش ارائه شده
    اشاره شود، اما در بخش نتیجه‌گیری به مسئله مورد بحث اشاره نشده است.
    \item نتایج پژوهش به صورت کیفی ارائه شده است بدین معنی که تفاوت کمی حاصل از اعمال ایده
    آن‌ها اشارده نشده است و صرفا سعی شده است با کلماتی نظیر «بیشتر» مفهوم منتقل شود.
    \item به مدل‌های استفاده شده در حین پژوهش اشاره نشده است و صرفا یک اشاره بسیار کلی به آن‌ها شده است.
\end{enumerate}

\subsection{نقاط قوت}

\begin{enumerate}
    \item روش‌های پیشنهاد شده در طول مقاله مجددا به صورت مختصر بیان شده است.
\end{enumerate}

\end{document}