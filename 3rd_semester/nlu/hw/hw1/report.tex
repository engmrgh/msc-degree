\documentclass[12pt, a4paper]{book}

\usepackage{fancyhdr}
\usepackage[left=4cm, right=4cm, top=4cm, bottom=4cm]{geometry}
\usepackage[utf8]{inputenc}
\usepackage[table]{xcolor}
\usepackage{hyperref}
\usepackage{amsmath}
\usepackage{enumitem}
\usepackage{graphicx}
\usepackage{amsfonts}
\usepackage{booktabs}
\usepackage{subcaption}
\usepackage[justification=centering]{caption}
\usepackage{xepersian}

\DeclareMathOperator*{\argmax}{argmax}
\DeclareMathOperator*{\argmin}{argmin}
\newcolumntype{L}{>{$}l<{$}} % math-mode version of "l" column type

\newcommand{\coursetitle}{فهم زبان طبیعی}
\newcommand{\doctitle}{تمرین اول}
\newcommand{\name}{محمدرضا غفرانی}
\newcommand{\studentno}{400131076}
\newcommand{\todaydate}{\today}

\settextfont{Sahel}
\setlatintextfont{Times Newer Roman}

\pagestyle{fancy}
\lhead{\textbf{\doctitle}}
\chead{\name}
\rhead{\todaydate}

\begin{document}

\begin{flushleft}
    \name \\
    \studentno \\
    \todaydate
\end{flushleft}

\begin{center}
    \huge
    \textbf{\coursetitle}
    \break
    \large
    \doctitle
\end{center}

% suppress the fancy header on the first page only
\thispagestyle{plain}

\section*{ساختار پروژه}

کد‌های پروژه در فایل \lr{code.py} در کنار این فایل ارسال شده است. برای اجرای کد‌ها باید فایل
\lr{main.py} را اجرا نمود. برای اجرای مدل \lr{transformer} از قطعه کد زیر

\begin{latin}
    \begin{center}
        python main.py --train\_dataset <TRAIN\_DATASET> --test\_dataset <TEST\_DATASET> --model transformer
    \end{center}
\end{latin}

و برای اجرای کد \lr{encoder-decoder} از قطعه کد زیر استفاده شود.

\begin{latin}
    \begin{center}
        python main.py --train\_dataset <TRAIN\_DATASET> --test\_dataset <TEST\_DATASET> --model encoder\_decoder
    \end{center}
\end{latin}

باقی تنظیمات مدل با استفاده از دستور \lr{--help} قابل دسترسی است.

\section*{پیش‌پردازش داده‌ها}

با توجه به آن که داده‌های متنی توسط شبکه قابل درک نیستند، ابتدا آن‌ها را به دنباله عددی تبدیل می‌کنیم.
برای انجام این تبدیل تمامی داده‌های آموزشی را پیمایش کرده و به هر کاراکتر در داده‌های گرافم
و فونم یک شناسه یکتا نسبت می‌دهیم. در قدم بعد ۳ کاراکتر نیز خود به مجموعه کاراکتر‌های فونم\LTRfootnote{phoneme}
و گرافم\LTRfootnote{grapheme}
اضافه کرده و به آن‌ها نیز شناسه یکتا نسبت می‌دهیم. یکی از کاراکتر‌های اضافه‌شده به منظور مشخص کردن
ابتدای گرافم/فونم، کاراکتر دیگر به منظور مشخص کردن انتهای گرافم/فونم و کاراکتر سوم به منظور مشخص
کردن پدینگ\LTRfootnote{padding} استفاده می‌شود.
در ادامه داده‌های آموزشی به نسبت ۸ به ۲ تقسیم شده و به ترتیب برای آموزش و ارزیابی مدل استفاده می‌شود.
جزئیات بیشتر روند انجام این فرآیند در فایل \lr{data.py} آورده شده است.

\section*{جزئیات شبکه‌های پیاده‌سازی شده}

\subsection*{مدل \lr{Encoder-Decoder}}

این شبکه از دو شبکه بازگشتی \lr{GRU} به عنوان کدگذار و کدگشا
برای تبدیل \lr{grapheme} به \lr{phoneme} استفاده می‌کند.
برای بهبود عملکرد این شبکه‌ها از یک لایه توجه نیز استفاده می‌شود. این لایه از آن جا که خروجی‌های کدگذار‌
در مراحل مختلف را با هم ترکیب کرده و در اختیار قسمت کدگشا قرار می‌دهد، در نتیجه عملکرد مدل را بهبود می‌بخشد.
برای رسیدن به عملکرد بهتر از شبکه بازگشتی دو جهته برای تولید خروجی کدگذار استفاده می‌کنیم.

در پیاده‌سازی انجام شده اندازه حالت مخفی شبکه \lr{GRU} برابر ۲۵۶ در نظر گرفته شده است. اندازه
تعبیه هر کاراکتر نیز ۲۵۶ در نظر گرفته شده است. برای جلوگیری از بیش‌برازش از \lr{dropout} با نرخ
۱۰ درصد استفاده شده است.

\subsection*{مدل \lr{Transformer}}

شبکه ترنسفورمری بر خلاف مدل قبلی حاوی به صورت متوالی عمل نمی‌کند و در نتیجه سریع‌تر عمل می‌کند.
این مدل تنها از مکانیزم توجه به همراه شبکه‌های جلورو برای تولید خروجی استفاده می‌کند. قسمت کدگذار این
شبکه از چندین بلوک تشکیل شده است که در هر بلوک ابتدا با استفاده از مکانیزم توجه به خود برای هر گرافم
تعبیه با کمک تعبیه‌های گرافم‌های مجاور ساخته می‌شود. سپس این بردار‌ها از شبکه جلو‌رو عبور داده می‌شوند.
اتفاق مشابهی در قسمت کدگشا می‌افتد با این تفاوت که در بین شبکه جلورو و قسمت توجه‌به‌خود، قسمتی برای
توجه به خروجی‌های کدگذار در نظر گرفته شده است. این قسمت خروجی تمامی کدگذار‌ها دسترسی دارد.

در پیاده‌سازی انجام شده نکات گفته‌شده در بالا رعایت شده است، در بین لایه‌ها در چند نقطه از
تکنیک \lr{dropout} استفاده شده است تا مدل روی داده‌های آموزشی بیش‌برازش نشود.
هر یک از قسمت‌های کدگذار و کدگشا حاوی ۸ لایه بلوک کدگذار و کدگشا است.
سایز لایه مخفی مدل برابر ۱۲۸ در نظر گرفته شده است. قسمت توجه‌به‌خود نیز حاوی
۸ سرِ توجه است.

\section*{نتایج}

نتایج اجرای مدل در جدول \ref{table2} آورده شده است. در جدول \ref{table1} نیز تعدادی از نمونه‌هایی که برای ارزیابی
عملکرد مدل استفاده شده به همراه خروجی‌هایی که این مدل‌ها تولید کرده‌اند آورده شده است.
همان‌طور که مشاهده می‌شود برخلاف انتظار مدل مبتنی بر شبکه بازگشتی عملکرد بهتری نسبت به مدل \lr{Transformer}
داشته است. هر دو مدل ۱۰ گام آموزش دیده و سپس ارزیابی شده‌اند. از لحاظ تعداد پارامتر‌ها نیز مدل‌ مبتنی بر
شبکه \lr{Transformer} حدود ۴ میلیون پارامتر دارد، در حالی که شبکه بازگشتی حدود ۲ میلیون پارامتر دارد.
این نتایج به ازای تعداد پارامتر‌های مختلف نیز تست شده است، و نتیجه مشابهی به دست آمده است.
در توجیه علت رویداد این اتفاق می‌توان گفت چون طول ورودی کم‌تر است در نتیجه
مدل‌های بازگشتی با مشکل فراموشی داده‌های اولیه مواجه نشده و در نتیجه توانسته‌اند عملکرد بهتری داشته باشند.

\begin{table}[!ht]
    \centering
    \caption{خطای مدل \lr{transformer} و \lr{encoder-decoder}}
    \label{table2}
   \begin{tabular}{l|l|l}
        & خطای کلمه       & خطای واج        \\ \hline
        \lr{tranformer}      & $0.61$     & $0.17$ \\
        \lr{encoder-decoder} & $0.45$     & $0.14$ \\
    \end{tabular}
\end{table}


\begin{latin}
\begin{table}[!ht]
    \centering
    \caption{\rl{جدول عملکرد مدل \lr{Transformer} و \lr{encoder-decoder} برای نمونه‌های ارزیابی}}
    \label{table1}
    \begin{tabular}{l|l|l|l}
        grapheme  & phoneme   & transformer & encoder-decoder \\ \hline
        \rl{آئینه‌خانه} & 'A'inexAne & 'A'inexAne   & 'A'inexAne       \\
        \rl{آئین‌نامه‌}  & 'A'innAme  & 'A'innAme    & 'A'innAme        \\
        \rl{آباده}     & 'AbAde     & 'AbAdee      & 'AbAde           \\
        \rl{آبتنی}     & 'Abtani    & 'Abtani      & 'Abtani          \\
        \rl{آبجوسازی}  & 'AbjosAzi  & 'AbjusAzi    & 'AbjusAzi        \\
        \rl{آبجی}      & 'Abji      & 'Abji        & 'Abeji           \\
        \rl{آبخوری}    & 'Abxori    & 'Abxori      & 'Abxori          \\
        \rl{آبدانک}    & 'AbdAnak   & 'AbdAnak     & 'AbdAnak         \\
        \rl{آبدره}     & 'Abdarre   & 'Abdaree     & 'Abdare          \\
        \rl{آبدندان}   & 'AbdandAn  & 'AbdandAn    & 'AbdandAn        \\
    \end{tabular}
\end{table}
\end{latin}



\end{document}